\chapter{Uvod}

\section{Svrha dokumenta}

Ovim dokumentom se, prije svega, nastoji dati opis funkcionalnosti softverskog rješenja koje nudi \textit{određene} mogućnosti kada je u pitanju proces čitanja .pdf dokumenata. Obzirom da je ovaj dokument namijenjen različitim klasama korisnika, on sadrži i opis funkcionalnosti sa različitim nivoima detalja. Te klase korisnika su:
\begin{itemize}
    \item Arhitekte sistema i razvojni programeri - s obzirom da su oni odgovorni za implementaciju softverskog rješenja, funkcionalnosti su opisane na visokom nivou detalja (detaljan opis funkcionalnih zahtjeva, nefunkcionalni zahtjevi, interfejsi i tipovi korisnika), kako bi proces \textit{razvoja} arhitekture (a i koda) ovog sistema bio što lakši.
    \item Krajnji korisnici (klijent) - krajnji korisnici (sistema) se ne brinu u velikoj mjeri o tome kako će ovaj sistem biti implementiran, odnosno, njihova interesovanja su uglavnom usmjerena ka tome \textbf{šta} će sistem raditi (da li je to ono što je njima trebalo), pa je kroz dokument dat i opis funkcionalnosti bez zalaženja u detalje.
\end{itemize}

\section{Opseg dokumenta}

Dokument sadrži specifikaciju softverskog rješenja \textit{Reader}, razvijeno od strane DevTech organizacije. Aplikacija \textit{Reader} olakšava korisnicima čitanje željenih .pdf dokumenata, nudeći im različite mogućnosti (kao što su izdvajanje dijelova dokumenta, označavanje omiljenih citata i slično). \\

Dokument je podijeljen u dva dijela - opis i konkretni zahtjevi. Kroz prvi dio funkcionalnosti proizvoda su opisana bez detaljisanja (namijenjen krajnjim korisnicima - klijentu), karakteristike korisnika aplikacije, ograničenja, pretpostavke i zavisnosti koje su vezane za upotrebu ove aplikacije te procedura koja će se slijediti ukoliko dođe do promjene zahtjeva.\\ 

U drugom dijelu dokumenta dat je detaljan opis funkcionalnih i nefunkcionalnih zahtjeva ovog sistema, njegovi interfejsi te atributi kvalitete (namijenjeno arhitektima i timu razvojnih programera).

\section{Standardi dokumentovanja}

Dokument je pisan u skladu sa IEEE 830-1998 standardom. Izrađen je kolaborativnim radom korištenjem alata \LaTeX.

\section{Akronimi i definicije}

\begin{itemize}
    \item \textbf{SRS} - Software Requirement Specification, dokument specifikacije zahtjeva softvera koji se razvija.
    \item \textbf{IEEE} - Institute of Electrical and Electronics Engineers, međunarodna profesionalna organizacija koja se bavi svim aspektima elektrotehnike, elektronike i računarstva.
    \item \textbf{IEEE 830-1998 standard} - skup procedura i pravila koje je preporučljivo slijediti pri izradi SRS-a.
    \item \textbf{Softversko rješenje} - programski kod zajedno sa pratećom dokumentacijom.
    \item \textbf{Funkcionalni zahtjevi} - prikaz aktivnosti koje bi sistem trebao ponuditi, odnosno, opis reakcije ssistema na ulaze i ponašanja sistema u različitim situacijama. Također, mogu specificirati i šta sistem ne bi trebao da radi.
    \item \textbf{Nefunkcionalni zahtjevi} - Ograničenja na servise i funkcije koje sistem nudi (vremenska ograničenja, ograničenja vezana za razvojni proces, ograničenja nametnuta standardima i slično).
    \item \textbf{Arhitektura sistema} - opis načina organizacije sistema koji ima značajan uticaj na osobine sistema kao što su performanse, sigurnost, dostupnost i slično.
    \item \textbf{Operativni sistem} - skup računarskih programa koji upravljaju hardverskim i softverskim resursima računara.
    \item \textbf{HTTPS} - Hyper Text Transfer Protocol Secure, protokol za razmjenu podataka između  \textit{browser}-a i web stranice na koju je korisnik konektovan, pri čemu je kompletna komunikacija šifrirana (eng. \textit{encrypted}).
    \item \textbf{JSON} - JavaScript Object Notation, format podataka koji se koristi za pohranjivanje (eng. \textit{storing}) i razmjenu (eng. \textit{exchange}) podataka.
    \item \textbf{REST} - REpresentational State Transfer, definiše skup funkcija kojima je moguće slati zahtjeve i primati odgovore koristeći HTTP protokol.
    \item \textbf{SPA} - Single Page Application, aplikacija koja nakon učitavanja odgovarajuće stranice, dinamički ažurira sadržaj stranice pri interakciji sa klijentom (bez učitavanja nove stranice sa servera).
    \item \textbf{Backup} - čuvanje rezervnih kopija podataka.
    \item \textbf{Server} - program (ili uređaj) koji pruža usluge drugim programima (ili uređajima), pri čemu se ti drugi programi (ili uređaji) nazivaju klijentima.
    \item \textbf{ReactJS} - JS biblioteka koja je korištena za razvoj korisničkog interfejsa (\textit{frontend}) aplikacije.
    \item \textbf{NodeJS} - JS biblioteka korištena za razvoj serverskog dijela (\textit{backend}) aplikacije.
    \item \textbf{\LaTeX} - sistem za pripremu i uređivanje dokumenata.
\end{itemize}

\section{Reference}

 \begin{itemize}
     \item Zakon o autorskim i srodnim pravima Bosne i Hercegovine
     \item Zakoni o autorskim i srodnim pravima postavljeni na Internacionalnoj Copyright Konvenciji (ICC) 1952. godine
     \item IEEE 803-1998 standard
 \end{itemize}